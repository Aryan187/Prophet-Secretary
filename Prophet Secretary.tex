\documentclass[12pt, letterpaper, twoside]{article}
\usepackage[left=2cm, right=2cm, top=1cm]{geometry}
\usepackage{amsmath,amsthm,amssymb}
\usepackage{xcolor}
%opening
\title{Prophet Secretary}

\begin{document}

\maketitle
Given that $F_{1}$,$F_{2}$,...$F_{n}$ are the initial distribution and $V_{1}$,$V_{2}$,...$V_{n}$ are the corresponding sampled values. For any t $\in [0,1] $, define:\\
$\theta(t)$ = The probability that nothing is selected till time t.\\
$q_{-i}(t)$ = The probability that nothing is selected till time t conditioned that $V_{i}$ arrives at time t.
$\alpha$ : [0,1] $\rightarrow$ [0,1] and $\beta$ : [0,1] $\rightarrow$ $\mathbb{R}$, both non-increasing such that,
\begin{align*}
Pr(\max_{i} \{V_{1},V_{2},...V_{n}\} \leq \alpha(t)) = \beta(t)
\end{align*}
COMMENT[AC]: Use \verb|\begin{lemma}...\end{lemma}| for lemmas. Use \verb|\Pr[...]| for probability. Symbols such as x, t, n need to be in math mode (\verb|$x$| etc.). Use \verb|\left(...\right)| when the content between the parentheses has more height than the usual text, eg.\ fractions.\\
$\textbf{LEMMA 1:}$ $\forall$ i $\in$ [n], t $\in$ [0,1]
\begin{align*}
q_{-i}(t) \geq \frac{\theta(t)}{1-t+\int_{0}^{t} Pr(V_{i} \leq \beta(x))  dx}
\end{align*}
COMMENT[AC]: Edit the proof according to our discussion.\\
\begin{proof}
\begin{align*}
\theta(t) &= \int_{0}^{1} Pr(T > t \mid t_{i} = x) dx\\
&= \int_{0}^{t} Pr(T > t \mid t_{i} = x) dx + \int_{t}^{1} Pr(T > t \mid t_{i} = x) dx
\end{align*}
For any x greater than t,
\begin{align*}
Pr(T > t \mid t_{i} = x) = Pr(T > t \mid t_{i} > t) \leq q_{-i}(t)
\end{align*}
For any x less than t and arbitary small $\epsilon$,
\begin{align*}
Pr(T > t \mid t_{i} = x) &= Pr(V_{i} \leq \beta(x)) (\text{Pr(Nothing is chosen before x) + Pr(Nothing is chosen in [x+$\epsilon$,t])})
\end{align*}
Since thresholds are non-increasing, for any $V_{k}$ appearing at $t_{k}$ $\in$ [x+$\epsilon$,t], Pr($V_{k} < \beta(t_{k})$) $\leq$ Pr($V_{k} < \beta(t_{k}-\epsilon))$. Therefore,
\begin{align*}
Pr(T > t \mid t_{i} = x) &\leq Pr(V_{i} \leq \beta(x)) (\text{Pr(Nothing is chosen before x) + Pr(Nothing is chosen in [x,t-$\epsilon$])})\\
&\leq Pr(V_{i} \leq \beta(x)) (q_{-i}(t))
\end{align*}
Putting the inequalities and rearranging gives the result.
\end{proof}
$\textbf{LEMMA 2:}$ $\forall$ t $\in$ [0,1/2]
\begin{align*}
\sum_{i=1}^{n} \frac{Pr(V_{i}>\beta(t))}{1-\int_{0}^{t} Pr(V_{i}>\beta(x)) dx} \geq \frac{Pr(\max_{i} \{V_{i}\} > \beta(t))}{1-tPr(\max_{i} \{V_{i}\} > \beta(0))}
\end{align*}
\begin{proof}
\begin{align*}
\sum_{i=1}^{n} \frac{Pr(V_{i}>\beta(t))}{1-\int_{0}^{t} Pr(V_{i}>\beta(x)) dx} &= \sum_{i=1}^{n} \frac{Pr(V_{i}>\beta(t))}{1-t+\int_{0}^{t} Pr(V_{i}\leq\beta(x)) dx}\\
&\geq \sum_{i=1}^{n} \frac{Pr(V_{i}>\beta(t))}{1-t+tPr(V_{i}\leq\beta(0))}\\
&=\sum_{i=1}^{n} \frac{1-F_{i}(\beta(t))}{1-t+tF_{i}(\beta(0))}
\end{align*}
Using the following inequality	 (Proof in the paper)
\begin{align*} 
\frac{1-F_{1}(\beta(t))}{1-t+tF_{1}(\beta(0))} + \frac{1-F_{2}(\beta(t))}{1-t+tF_{2}(\beta(0))} \geq \frac{1-F_{1}(\beta(t))F_{2}(\beta(t))}{1-t+tF_{1}(\beta(0))F_{2}(\beta(0))} 
\end{align*}  
and repeating it n times, we get the required result.
\end{proof}
For the competitive ratio of 0.669, we define $\alpha_{1}, \alpha_{2},...\alpha_{m}$ and set $\alpha(t) = \alpha_{ceil(tm)}$\\
We also define a non-decreasing function g, for $t \in\ [0,1]$,
$$g_{p}(t) = \begin{cases}
\frac{1}{1-t(1-p)}  \quad \,; t \leq 1/2 \\
\frac{2}{1+p} \quad \quad \quad; t > 1/2
\end{cases}
$$
Using the result from the last section, if z is the reward, for any $t \in [0,1]$, we can write,\\
\begin{align*}
Pr(z > \beta(t)) &= \frac{1-\theta(t)}{1-\alpha(t)} (1-\alpha(t)) + \sum_{i \in [n]} Pr(V_{i} > \beta(t)) \int_{t}^{1} q_{-1}(x) dx
\end{align*}
Using the bound for $\theta(t)$ derived in the last part, the first term can be written as,
\begin{align*}
 \frac{1-\theta(t)}{1-\alpha(t)} (1-\alpha(t)) &\geq \frac{\int_{0}^{t} 1 - \alpha(x) dx}{1-\alpha(t)} (Pr(\max_{i} \{V_{i}\} > \beta_{t})
\end{align*}
For the second term, first we use Lemma 1 to write,
\begin{align*}
q_{-i}(x) \geq \frac{\theta(x)}{1-\int_{0}^{x} Pr(V_{i} > \beta(y))  dy}
\end{align*}
Then interchanging the order of sums,
\begin{align*}
\sum_{i \in [n]} Pr(V_{i} > \beta(t)) \int_{t}^{1} q_{-1}(x) dx \geq \int_{t}^{1} \theta(x) \sum_{i \in [n]} \frac{Pr(V_{i} > \beta(t))}{1 - \int_{0}^{x} Pr(V_{i} > \beta(y)) dy} dx
\end{align*}
Now, using Lemma 2, for x $\leq$ 1/2
\begin{align*}
\sum_{i \in [n]} \frac{Pr(V_{i} > \beta(t))}{1 - \int_{0}^{x} Pr(V_{i} > \beta(y)) dy} &\geq \frac{Pr(\max_{i} \{V_{i}\} > \beta(x))}{1-xPr(\max_{i} \{V_{i}\} > \beta(0))}\\
&= g_{\alpha(0)} (x) Pr(\max_{i} \{V_{i}\} > \beta(x))
\end{align*}
Now, for x $>$ 1/2, it can be observed that \\
\begin{align*}
\sum_{i \in [n]} \frac{Pr(V_{i} > \beta(t))}{1 - \int_{0}^{x} Pr(V_{i} > \beta(y)) dy} &\geq \frac{Pr(\max_{i} \{V_{i}\} > \beta(x))}{1-\frac{1}{2}Pr(\max_{i} \{V_{i}\} > \beta(0))} \\
&= g_{\alpha(0)} (x) Pr(\max_{i} \{V_{i}\} > \beta(x))
\end{align*}
Finally, we divide x according to the m thresholds, and use the bound for $\theta(t)$ derived in the last section to get the final result.
\begin{align*}
\int_{t}^{1} \theta(x) g_{\alpha(0)} (x) dx &= \sum_{i = tm}^{m} \int_{i/m}^{(i+1)/m} \theta(x) g_{\alpha(0)} (x) dx\\
&\geq \sum_{i = tm}^{m} \int_{i/m}^{(i+1)/m} \theta(i/m) Pr(V_{time=x} < \beta_{i})^{x - i/m} g_{\alpha(0)} (i/m) dx\\
&\geq \sum_{i = tm}^{m} g_{\alpha(0)}(i/m) e^{\int_{0}^{i/m} ln(\alpha(y)) dy} \int_{i/m}^{(i+1)/m} \alpha_{i}^{x - i/m}  dx\\
&= \sum_{i = tm}^{m} g_{\alpha(0)}(i/m) (\prod_{k=1}^{i} \alpha_{k})^{1/m} \int_{i/m}^{(i+1)/m} \alpha_{i}^{x - i/m}  dx\\
&= \sum_{i = tm}^{m} g_{\alpha(0)}(i/m) (\prod_{k=1}^{i} \alpha_{k})^{1/m} \frac{1 - \alpha_{i} ^ {1/m}}{-ln(\alpha_{i})}
\end{align*}
COMMENT[AC]: I don't understand the first inequality onward. If you do, please add explanations. What is the meaning of $Pr(V_{time=x} < \beta_{i})$?\\
Hence, we proved that:
\begin{align*}
Pr(z > \beta(t)) \geq (\frac{\int_{0}^{t} 1 - \alpha(x) dx}{1-\alpha(t)} + \sum_{i = tm}^{m} g_{\alpha(0)}(i/m) (\prod_{k=1}^{i} \alpha_{k})^{1/m} \frac{1 - \alpha_{i} ^ {1/m}}{-ln(\alpha_{i})}) Pr(\max_{i} \{V_{i}\} > \beta_{t})
\end{align*}
Where optimization of values for $\alpha$ lead to a competitive ratio of 0.669
\end{document}
