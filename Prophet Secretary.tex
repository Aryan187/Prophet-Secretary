\documentclass[10pt, letterpaper, twoside]{article}
\usepackage[left=2cm, right=2cm, top=1cm]{geometry}
\usepackage{amsmath,amsthm,amssymb}
\usepackage{xcolor}
\newtheorem{theorem}{Theorem}[section]
\newtheorem{corollary}{Corollary}[theorem]
\newtheorem{lemma}[theorem]{Lemma}
%opening
\title{Prophet Secretary}

\begin{document}
	
	\maketitle
	Given that $F_{1}$,$F_{2}$,...$F_{n}$ are the initial distribution and $V_{1}$,$V_{2}$,...$V_{n}$ are the corresponding sampled values. For any $t$ $\in [0,1] $, define:\\
	$\theta(t)$ = The probability that nothing is selected till time $t$.\\
	$q_{-i}(t)$ = The probability that nothing is selected till time $t$ conditioned that $V_{i}$ arrives at time $t$.
	$\alpha$ : $[0,1]$ $\rightarrow$ $[0,1]$ and $\beta$ : [0,1] $\rightarrow$ $\mathbb{R}$, both non-increasing such that,
	\begin{align*}
		\Pr[\max_{i} \{V_{1},V_{2},...V_{n}\} \leq \alpha(t)] = \beta(t)
	\end{align*}
\begin{lemma} $\forall i \in [n], t \in [0,1]$
	\begin{align*}
		q_{-i}(t) = \frac{\theta(t)}{1-t+\int_{0}^{t} \Pr[V_{i} \leq \beta(x)]  dx}
	\end{align*}
\end{lemma}
	\begin{proof}
		\begin{align*}
			\theta(t) &= \int_{0}^{1} \Pr[T > t \mid t_{i} = x] dx\\
			&= \int_{0}^{t} \Pr[T > t \mid t_{i} = x] dx + \int_{t}^{1} \Pr[T > t \mid t_{i} = x] dx
		\end{align*}
		For any $x$ greater than $t$,
		\begin{align*}
			\Pr[T > t \mid t_{i} = x] = \Pr[T > t \mid t_{i} > t] = q_{-i}(t)
		\end{align*}
		For any $x$ less than $t$,
		\begin{align*}
			\Pr[T > t \mid t_{i} = x] &= \Pr[V_{i} \leq \beta(x)] (\text{Pr[Nothing is chosen before $x$ $\mid$ $V_{i}$ does not come before $x$]} \\
			&\quad \quad \quad \quad \quad \text{+ Pr[Nothing is chosen from $x$ to $t$ $\mid$ $V_{i}$ does not come between $x$ and $t$]})\\
			&= \Pr[V_{i} \leq \beta(x)](\text{Pr[Nothing is chosen before $t$ and $V_{i}$ does not come before $t$]})\\
			&= \Pr[V_{i} \leq \beta(x)] (q_{-i}(t))
		\end{align*}
COMMENT[AC]: The above calculations look wrong to me, although the end result is correct. How about the following?
Simple observation: $\Pr[\text{nothing chosen before time }t \mid V_i\leq\beta(t_i)]=\Pr[\text{nothing chosen before time }t \mid t_i>t]=q_{-i}(t)$. Therefore,
\begin{eqnarray*}
\Pr[T > t \mid t_{i} = x] & = & \Pr[V_{i} \leq \beta(x)] \times \Pr[\text{nothing chosen before time }t \mid V_{i} \leq \beta(x) \wedge t_{i} = x]\\
 & = & \Pr[V_{i} \leq \beta(x)] \times \Pr[\text{nothing chosen before time }t \mid V_{i} \leq \beta(t_i) \wedge t_{i} = x]\\
 & = & \Pr[V_{i} \leq \beta(x)] \times \Pr[\text{nothing chosen before time }t \mid V_{i} \leq \beta(t_i)]\\
 & = & \Pr[V_{i} \leq \beta(x)] \times q_{-i}(t)
\end{eqnarray*}
Basically if $V_{i} \leq \beta(t_i)$ then we are sure $i$ is not going to stop the algorithm. Then the stopping time is independent of its arrival time $t_i$. Makes sense?

		Putting the inequalities and rearranging gives the result.
	\end{proof}
\begin{lemma}
$\forall t \in [0,1/2]$
	\begin{align*}
		\sum_{i=1}^{n} \frac{\Pr[V_{i}>\beta(t)]}{1-\int_{0}^{t} \Pr[V_{i}>\beta(x)] dx} \geq \frac{\Pr[\max_{i} \{V_{i}\} > \beta(t)]}{1-t\Pr[\max_{i} \{V_{i}\} > \beta(0)]}
	\end{align*}
\end{lemma}
	\begin{proof}
		\begin{align*}
			\sum_{i=1}^{n} \frac{\Pr[V_{i}>\beta(t)]}{1-\int_{0}^{t} \Pr[V_{i}>\beta(x)] dx} &= \sum_{i=1}^{n} \frac{\Pr[V_{i}>\beta(t)]}{1-t+\int_{0}^{t} \Pr[V_{i}\leq\beta(x)] dx}\\
			&\geq \sum_{i=1}^{n} \frac{\Pr[V_{i}>\beta(t)]}{1-t+t\Pr[V_{i}\leq\beta(0)]}\\
			&=\sum_{i=1}^{n} \frac{1-F_{i}(\beta(t))}{1-t+tF_{i}(\beta(0))}
		\end{align*}
		Using the following inequality	 (Proof in the paper)
		\begin{align*} 
			\frac{1-F_{1}(\beta(t))}{1-t+tF_{1}(\beta(0))} + \frac{1-F_{2}(\beta(t))}{1-t+tF_{2}(\beta(0))} \geq \frac{1-F_{1}(\beta(t))F_{2}(\beta(t))}{1-t+tF_{1}(\beta(0))F_{2}(\beta(0))} 
		\end{align*}  
		and repeating it n times, we get the required result.
	\end{proof}
	For the competitive ratio of 0.669, we define a non-decreasing function $g$, for $t \in\ [0,1]$,
	$$g_{p}(t) = \begin{cases}
	\frac{1}{1-t(1-p)}  \quad \,; t \leq 1/2 \\
	\frac{2}{1+p} \quad \quad \quad; t > 1/2
	\end{cases}
	$$
	Using the result from the last section, if $z$ is the reward, for any $t \in [0,1]$, we can write,\\
	\begin{align*}
		\Pr[z > \beta(t)] &= \frac{1-\theta(t)}{1-\alpha(t)} (1-\alpha(t)) + \sum_{i \in [n]} \Pr[V_{i} > \beta(t)] \int_{t}^{1} q_{-1}(x) dx
	\end{align*}
	Using the bound for $\theta(t)$ derived in the last part, the first term can be written as,
	\begin{align*}
		\frac{1-\theta(t)}{1-\alpha(t)} (1-\alpha(t)) &\geq \frac{\int_{0}^{t} 1 - \alpha(x) dx}{1-\alpha(t)} (\Pr[\max_{i} \{V_{i}\} > \beta_{t}]
	\end{align*}
	For the second term, first we use Lemma 0.1 to write,
	\begin{align*}
		q_{-i}(x) \geq \frac{\theta(x)}{1-\int_{0}^{x} \Pr[V_{i} > \beta(y)]  dy}
	\end{align*}
	Then interchanging the order of sums,
	\begin{align*}
		\sum_{i \in [n]} \Pr[V_{i} > \beta(t)] \int_{t}^{1} q_{-1}(x) dx \geq \int_{t}^{1} \theta(x) \sum_{i \in [n]} \frac{\Pr[V_{i} > \beta(t)]}{1 - \int_{0}^{x} \Pr[V_{i} > \beta(y)] dy} dx
	\end{align*}
	Now, using Lemma 2, for $x$ $\leq$ $1/2$
	\begin{align*}
		\sum_{i \in [n]} \frac{\Pr[V_{i} > \beta(t)]}{1 - \int_{0}^{x} \Pr[V_{i} > \beta(y)] dy} &\geq \frac{\Pr[\max_{i} \{V_{i}\} > \beta(x)]}{1-x\Pr[\max_{i} \{V_{i}\} > \beta(0)]}\\
		&= g_{\alpha(0)} (x) \Pr[\max_{i} \{V_{i}\} > \beta(x)]
	\end{align*}
	Now, for $x$ $>$ $1/2$, it can be observed that \\
	\begin{align*}
		\sum_{i \in [n]} \frac{\Pr[V_{i} > \beta(t)]}{1 - \int_{0}^{x} \Pr[V_{i} > \beta(y)] dy} &\geq \frac{\Pr[\max_{i} \{V_{i}\} > \beta(x)]}{1-\frac{1}{2}\Pr[\max_{i} \{V_{i}\} > \beta(0)]} \\
		&= g_{\alpha(0)} (x) \Pr[\max_{i} \{V_{i}\} > \beta(x)]
	\end{align*}
	Now, using the bound for $\theta(x)$ derived in the last section,
	\begin{align*}
		\theta(x) \geq e^{\int_{0}^{x} \ln(\alpha(y)) dy}
	\end{align*}
	We get, for each $t \in [0,1]$:
	\begin{align*}
		\Pr[z > \beta(t)] \geq \min_{t}\left(\frac{\int_{0}^{t} 1 - \alpha(x) dx}{1-\alpha(t)} + \int_{t}^{1} e^{\int_{0}^{x} \ln(\alpha(y)) dy} g_{\alpha(0)}(x) dx \right) \Pr[\max_{i} \{V_{i}\} > \beta_{t}]
	\end{align*}
	Where optimization of values for $\alpha$ lead to a competitive ratio of 0.669
\end{document}
